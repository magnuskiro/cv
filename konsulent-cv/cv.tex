\documentclass[11pt,a4paper,sans]{moderncv}

%%-----------------------------------------------------------------------------
%% Required package, everything inside this is for Norwegian/English language
%%support
\usepackage{ifthen}
\usepackage[norsk,english]{babel}

%% TeX function that generates the language commands.
\def\localedef#1#2{
\ifthenelse{ \equal{\locale}{#1} }{
  \selectlanguage{#2}
  \expandafter\def\csname#1\endcsname ##1{##1}
  }{
  \expandafter\def\csname#1\endcsname ##1{}
  }
}

%%Controls the language output
\providecommand\locale{no}

%% Language commands.
\localedef{no}{norsk}
\localedef{en}{english}
%%-----------------------------------------------------------------------------

%% ModernCV themes
\moderncvtheme[green]{classic}
%\moderncvtheme[black]{classic}
%\moderncvtheme[black]{banking}
\renewcommand{\familydefault}{\sfdefault}
%\nopagenumbers{}

%% Character encoding
\usepackage[utf8]{inputenc}

%% Adjust the page margins
\usepackage[scale=0.90]{geometry}

%%URL support
\usepackage{hyperref}

%%% Classic style edits. 
%%% http://ctan.uib.no/macros/latex/contrib/moderncv/moderncvstyleclassic.sty
%% change size of contact info. 
% fonts
\renewcommand*{\addressfont}{\large\mdseries\slshape}
% styles
\renewcommand*{\addressstyle}[1]{{\addressfont\textcolor{color1}{#1}}}
%% set size of left column. 
\setlength{\hintscolumnwidth}{3.1cm}

%% Personal data
\firstname{Magnus Løken}
\familyname{Kirø}
%\address{Postboks 146, 
%\en{Apartment}\no{Leilighet} 
%}{3166 Tolvsrød,
%\en{Norway}\no{Norge}}
%\phone[mobile]{+47~91~74~85~66}
%\email{magnuskiro@gmail.com}
%\homepage{magnuskiro.no}
%\social[linkedin]{magnuskiro}
%\social[twitter]{magnuskiro}
%\social[github]{magnuskiro}
\extrainfo{\textdagger 15.03.1990}
%\photo[100pt][1pt]{test.jpg}

% title is the current position you have. 
\title{Web-/Systemutvikler}
%%------------------------------------------------------------------------------
%% Content
%%------------------------------------------------------------------------------
\begin{document}
\makecvtitle{}

\subsection{\no{Sammendrag}}
% Sammendrag av deg som kosnulent.
\cvline{}{
\no{
%Her skal du skrive en egen beskrivelse av deg. 
%%%%%----------------------------------------------
%1. Innledning: Beskriv din kompetanse  med to-tre setninger, referere til din
%erfaring og  kompetens, bransch eller projekt. 
Magnus har fem års erfaring med IT og ledelse fra frivillige organisasjoner i
tilknytning til Studentersamfundet i Trondhjem.
Kjernekompetansen ligger på web-/systemutvikling, samt IT-support. 
Ledelse, og organisasjonsarbeid har han også jobbet med. Som IT-sjef i Studentmediene
i Trondheim AS har han også erfaring med ledelse, prosjektarbeid, og strategi.  
%%%%%----------------------------------------------
\newline
%2. Utdanning:  Utdannet innen… retning, eventuelle sertifisernger..
Magnus har fullført master i informatikk på Norges Teknisk-Naturvitenskaplige
Universitet med informasjonsforvaltning som studieretning. 
Utover mastergraden har Magnus også studert fag innenfor psykologi, og ledelse.
%%%%%----------------------------------------------
\newline
%3. Hva utmerker deg/ Hvordan fungerer du i prosjekt?
% beskriv hva erfaringene har gitt deg. Personlige egenskaper som er verdt å
% framheve for kunden. 
Som leder i prosjektarbeid er Magnus en pådriver som holder oversikt og vet litt om alt. Han
er løsningsorienter, og har fokus på målsetningene. Magnus er en omgjengelig
person som er lett å prate med. Med bred kompetanse er Magnus en svært allsidig
konsulent som er interessert i å påta seg et stort spekter av oppgaver. 
%%%%%----------------------------------------------
% ”Det som utmerker seg ved Anna i prosjekt er..”  hvordan jobber du i prosjekt,
%lederrolle, drivkraft, problemløser, etc …”.
%Bruk malen som inspirasjon og skriv en egen presentasjon ut i fra bredvid som
%inspiration och skriv en egen presentation utifra samme stil.
}
}

% Beskrivende hovedtrekk ved deg.
\subsection{\no{Hovedkvalifikasjoner /
Spesialkunnskaper}}\label{specialQualifications}
\no{
%Her skriver du dine fremste kompetanseområder. 
\cvline{}{
Web-/Systemutvikling
}
\vspace{.2cm}
\newline
%%-----------------------------------------------------------------------------
%Løft frem tre erfaringer du er mest stolt av som karakteriserer deg litt
%ekstra. (punktliste)
\cvline{Fremhevet erfaring}{
\begin{itemize}
\item IT-sjef i Studentmediene i Trondheim AS. \newline
Ledelse, personalansvar, og prosjektansvar. 
%%-----------------------------------------------------------------------------
\item Maskinist, Under Dusken. \newline
IT-ledelse, IT-drift, systemutvikling. 
%%-----------------------------------------------------------------------------
\item Systemutvikler, iNic AS \newline
Lærte programmering i Perl. Laget webservice som registrer domener hos NORID. 
\end{itemize}
}
}

\section{\no{Web-/Systemutvikler}}
\subsection{\no{Teknisk Profil}}
% Under denne seksjonen og de følgende skal du liste samtlige teknikker du har
% jobbet med, uansett hvor stor eller liten erfarening du har innen aktuell
% teknikk. Ikke vær redd for å ta med noe selv om du bare har sittet og plukket
% litt med teknikken. Dersom du har jobbet med en teknikk er det kunnskap,
% uansett om du er ekspert innen emnet eller ei. Det kreves ikke at du fått
% betalt for å lære deg noe. Kompetanse som kommer fra egenintresse er lik
% relevant.
\newline

\cvline{Arbeidsområde}{
Webutvikling, Systemutvikling, Systemintegrasjon, Kravspessifisering,
Systemarkitektur, Databasedesign, Personaledelse, Prosjektledelse,
Organisasjonsledelse, IT-Rådgivning, IT-strategi, IT-drift, IT-infrastruktur,
IT-support.
}
%%-----------------------------------------------------------------------------
\cvline{Utviklingsspråk og rammeverk}{
Java, Scala, Pythnon, PHP, C, C++, GO, Perl, Bash, Shell, Lua, JavaScript,
Assembly, Hibernate, \LaTeX, 
Net::EPP::Client, Spring, Flask, jQuery, Knockout.js, Angular.js, D3.js, Bootstrap, Grunt, 
nltk, JQuery, PLayFramework, 
}
%%-----------------------------------------------------------------------------
\cvline{Web- og mobilutvikling}{
XML, HTML5, CSS3, JSON, Web Services, Apache VelocityViews,
}
%%-----------------------------------------------------------------------------
\cvline{Arkitektur og Design}{
BPMN, UML, MVC 
}
%%-----------------------------------------------------------------------------
\cvline{Metodikk}{
Agile, Scrum, Waterfall, ITIL Foundation, 
}
%%-----------------------------------------------------------------------------
\cvline{Databaser}{
PostgreSQL, MySQL, HSQL
}
%%-----------------------------------------------------------------------------
\cvline{Test}{
Junit
}
%%-----------------------------------------------------------------------------
\cvline{Produkter/Miljøer}{
Intellij IDEA, Pycharm, Eclipse, Linux Mint, Debian, Ubuntu, Ubuntu
server, Windows 7, Awesome, 
MediaWiki, Joomla, Wordpress, Confluence, Android SDK,
Google analytics, SSH, Subversion, Git, GIMP, Vim, OwnCloud, Apache2,
Bower, Maven, Ant, Aptitude,  
phpMyAdmin, phpLDAPadmin, LDAP, Jenkins, Microsoft Office, Microsoft Outlook,
Sonatype Nexus,  
}
%%-----------------------------------------------------------------------------
\cvline{Annet}{
}
%%-----------------------------------------------------------------------------
% Make sure this section is the same as \ref{specialQualifications}
\cvline{Spesialkunnskaper}{
Web-/Systemutvikling  
}
%%-----------------------------------------------------------------------------

\vspace{.2cm}
\subsection{\no{Annet}}
%[- Her skal du skrive alle relevante utdanninger og sertifiseringer du har tatt.
%Samt språk du kan.]
\subsubsection{Utdanning}\newline
\cvline{08.2014 - 06.2019}{
\en{
\large{Norwegian University of Science and Technology, individual courses}
% main content - En
\newline
Management, psychology, and IT.
}\no{
\large{Norges Teknisk-Naturvitenskapelige Universitet, individuelle studier}
% Hovedinnold - No
\newline
Ledelse, psykologi, og IT.
}}
%%-----------------------------------------------------------------------------

\cvline{08.2012 - 06.2014}{
\en{
\large{NTNU, master in computer science}
% main content - En
\newline
Thesis in \textit{Artificial Intelligence} and \textit{Data and Information
Management}; '\textit{Tweet Sentiment, Sentiment Trend, and a Comparison with
Financial Trend Indicators.}'
}\no{
\large{NTNU, master informatikk}
% Hovedinnold - No
\newline
Masteroppgave innen \textit{Data- og informasjonsforvaltning}; '\textit{Tweet Sentiment, Sentiment Trend, and a Comparison
with Financial Trend Indicators.}'
}}
%%-----------------------------------------------------------------------------

\cvline{08.2009 - 06.2012}{
\en{
\large{NTNU, bachelor in computer science}
% main content - En
\newline
System development and artificial intelligence have been in focus during the
degree. The results of the bachelor thesis, \textit{Role-based Quality of
Service for Web
Services}, was
published by FFI(Norwegian Defence Research Establishment) in \textit{Globecom
Workshops}.
}\no{
\large{NTNU, bachelor informatikk}
% Hovedinnold - No
\newline
Systemutvikling og kunstig intelligens har vært mine fokusområder under
bachelorgraden. 
Resultatene fra bacheloroppgaven, \textit{Role-based Quality of Service for Web
Services}, er publisert av Forsvarets
Forskningsinstitutt i \textit{Globecom Workshops}.
}}
%%-----------------------------------------------------------------------------

\cvline{08.2006 - 06.2009}{
\en{
\large{Greveskogen high school, Tønsberg}
% main content - En
\newline
Focus on math, physics and IT.
}\no{
\large{Greveskogen Vidregående Skole, Tønsberg}
% Hovedinnold - No
\newline
Fokus på matte, fysikk og IT.
}}
%%-----------------------------------------------------------------------------

%\cvline{08.2003 - 06.2006}{
%\en{
%\large{Preseterød Secondary School, Tønsberg}
%% main content - En
%}\no{
%\large{Presterød Ungdomskole, Tøsnberg} 
%}
%% Hovedinnold - No
%}}
%%-----------------------------------------------------------------------------

%%-----------------------------------------------------------------------------
\vspace{.2cm}
\subsubsection{Kurs og Sertefiseringer}\newline
%\cvline{}{Ingen}
% TODO: add file for certifications. 
\cvline{07.02.2012}{
\en{
6 hour course in accounting and budgeting for volunteer
organizations.
}\no{
Regnskap. 
6 timers kurs i regnskap og budsjettering for frivillige organisasjoner.
}}


%%-----------------------------------------------------------------------------
\vspace{.2cm}
\subsubsection{\en{Languages}\no{Språk}}\newline
\en{
\cvline{}{
Norwegian – Native
\newline
English – Fluent
\newline
German – Limited working proficiency
}}\no{
\cvline{}{
Norsk – Morsmål
\newline
Engelsk – Flytende skriftlig og muntlig
\newline
Tysk – Elementært skriftlig og muntlig
}}
%%-----------------------------------------------------------------------------



%%-----------------------------------------------------------------------------

% Separation here summary first, experience after this. 
\newpage

% list / table of employees
% list / table of employees
\section{\no{Arbeidserfaring}\en{Work Experience}}
%[- Tabell med oversikt over arbeidsgivere og år for ansettelse.]

\cvline{}{
\begin{table}
\label{tbl:label}
%\caption{The text that shows in listOfTables}
\begin{tabular}{ p{5.5cm} p{3.5cm} p{6.5cm} }
Arbeidsgiver & År & Stilling \\
\hline
Næringslivsnettverket KiD & Q3.2014 - Q2.2015 & Systemutvikler \\
%JavaZone& Q3.2014 & Salvakt \\
Studentmediene i Trondheim AS & Q2.2013 - Q4.2014 & 
\parbox[t]{6.5cm}{IT-sjef, Systemutvikler,\\ Intern områdeansvarlig} \\
%Samfundets Interne Teater& Q1.2013 - Q4.2014 & Kulissebygger \\
Academic Work & Q4.2012 - Q2.2013 & Konsulent \\
Under Dusken& Q1.2010 - Q2.2013 & 
\parbox[t]{6.5cm}{Maskinist, Systemutvikler,\\ Intern områdeansvarlig} \\
iNic AS  & Q2.2012 - Q3.2012 & Systemutvikler \\
%Slottsfjell-festivalen& Q2.2010 \& Q2.2011 & Frivillig \\
%Proffcom & Q2.2010 - Q3.2010 & Kundeservice \\
%Tønsberg Ju Jitsu Klubb & Q3.2008 - Q2.2009 & Instruktør \\
\end{tabular}
\end{table}
}

\subsection{\no{Beskrivelse av Arbeidsgivere}}
%[- Beskrivelse av gitt arbeidsgiver. Hvem, hva, hvorfor.]

\cvline{Næringslivsnettverket KiD}{
KiD står for Kommunikasjons-, Informasjons- og Datateknologi. 
Næringslivsnettverket jobber for å bedre kontakten mellom studenter på NTNU og
bedrifter i bransjen. 
KiD er et sammarbeid mellom studenter, ansatte ved NTNU, og
sammarbeids-bedriftene. 
}

%\cvline{JavaZone}{
%JavaZone er Nord-Europas største, årlig konferase for JAva-utvikløere. Den
%foregikk fra 9.11. september i Oslo Spektrum, og samlet 2500 deltakere til 12
%workshops, 60 lyntaler og 90 foredrag. En konferanse på denne størrelse, som
%blir arrangert på frivillig basis, vil ikke være mulig å gjennomføre uten et
%sterkt engasjement fra fagmiljøet.
%}

\cvline{Studentmediene i Trondheim AS}{
Studentmediene i Trondheim er drevet på frivillig basis av studenter i
Trondheim, og består av papiravisen Under Dusken, nettavisen Dusken.no,
studentradioen Radio Revolt, og Student-TV. I tillegg har Studentmediene egne
avdelinger for blant annet marked, salg, økonomi og IT. Vi drifter alle
produktene selv, samt utvikler nye produkter som iBok, nettside for kjøp og
salg av pensumbøker, og Barteguiden, en app som gir oversikt over arrangementer
i Trondheim. I tillegg til å være AS er Studentmediene en del av
Studentersamfundet i Trondhjem.
}

%\cvline{Samfundets Interne Teater}{
%SIT er teatergruppa på Studentersamfundet i Trondhjem. Dette er en gjeng på ca
%70 personer som har innslag på en rekke av styrets lørdagsmøter, og som også
%setter opp egne produksjoner. Under UKA har SIT ansvaret for alle
%teaterproduksjonene, ikke minst UKE-revyen.
%Gruppa ble offisielt stiftet 3. desember 1910.
%}

\cvline{Academic Work}{
Academic Work er et bemannings- og konsulentselskap som spessialiserer seg på
studenter og nyutdannede. 
}

\cvline{Under Dusken}{
Studentavisa i Trondheim. Under Dusken er tilknyttet Studentersamfundet i Trondhjem. Avisa ble
stiftet i 1914, og er skandinavia sin eldste studentavis. Avisa kom med innhold på nett
i 1994 og har siden da hatt nettavis og datamedarbeidere.
}

\cvline{iNic AS}{
iNic AS / FastName.no er en ledende NORID/Uninett-akkreditert registrar,
kapabel til å registrere toppnivådomenenavn over store deler av verden.
\newline
Vi leverer tjenester innenfor domeneregistrering, webhotell, epost og
løsninger innen design/programmering. Pr. oktober 2010 betjener vi over 15000
kunder med til sammen over 39000 domenenavn. Vi er med dette en av landets
største registrarer av domenenavn.
}

%\cvline{Proffcom}{
%}

%\cvline{Tønsberg Ju Jitsu Klubb}{
%}




% detailed descriptions of positions and projects.
\section{\no{Prosjekterfaring}\en{Qualifications}}
%[- Detaljerte beskrivelser av arbeidet som er utført hos de forskjellige
%arbeidsgiverene.]

%[- Her under lister du opp all prosjekterfaring. Ikke glem deltidsstillinger og
%ekstrajobber. 
%Del opp lengre prosjekt eller stillinger etter ulike prosjekt du jobbet  i
%og/eller etter ulike roller du hatt i prosjektet slik at det blir flere poster,
%da blir din kompetanse mer lettlest.]



%%%%KID
\subsection{\no{Systemutvikler}}
% description of this position. 
\cvline{Q3.2014 - Q2.2015}{Næringslivsnettverket KID}
\cvline{}{
Magnus har utviklet http://kid.item.ntnu.no/ videre. Han har jobbet mest med backend
programmernig og drift av webserveren. Magnus har også
jobbet med frontend-utvikling. 
}
%%-----------------------------------------------------------------------------
% Technologies used in this project.  
\cvline{Teknikker}{
Java, Scala, HTML5, CSS3, JavaScript, PlayFramework, Knockout.js, jQuery, MySQL 
}
%%-----------------------------------------------------------------------------
% Environments used in this project. 
\cvline{Miljøer}{
Linux, Ubuntu Server, Intellij IDEA, phpMyAdmin, 
}
%%-----------------------------------------------------------------------------

%%%%SM - Barweb
\subsection{\no{Systemutvikler}}
\cvline{Q3.2014 - Q4.2014}{Studentmediene i Trondheim AS}
\cvline{}{
Magnus har startet ny-utviklingen av et internt system fra Under Dusken og
fullførte alpha versjonen.
Systemet holder oversikt over transaksjoner og regninger, fra sosiale
sammenkomster på Samfundet.
Prosjektet ble overlevert ny prosjekleder i god
stand, og med klare mål og oppgaver for videre utvikling.
}
%%-----------------------------------------------------------------------------
% Technologies used in this project.  
\cvline{Teknikker}{
Java, HTML5, CSS3, JavaScript, Spring, Angular.js, XML
}
%%-----------------------------------------------------------------------------
% Environments used in this project. 
\cvline{Miljøer}{
Linux, Ubuntu Server, Intellij IDEA, Jenkins
}
%%-----------------------------------------------------------------------------

%%%%SM - IT-drift
\subsection{\no{IT-drift}}
\cvline{Q2.2013 - Q4.2014}{Studentmediene i Trondheim AS}
\cvline{}{
Magnus har gjennom sin tid i Studentmediene vært med i IT-drifts-komiteen.
Komiteen har ansvaret for servere, arbeidsstasjoner, nettverk, og printere.
IT-drift jobber til daglig tett opp mot IT-utvikling. Magnus har hatt ansvar
for de gamle serveren til Under Dusken og utfasing av disse og han har hatt en
rådgivende rolle i komiteen. Magnus har tatt del i oppsett og planlegging av ny
infrastruktur.
}
%%-----------------------------------------------------------------------------
% Technologies used in this project.  
\cvline{Teknikker}{
}
%%-----------------------------------------------------------------------------
% Environments used in this project. 
\cvline{Miljøer}{
Linux, Ubuntu Server, Debian, Intellij IDEA, MediaWiki,
}
%%-----------------------------------------------------------------------------

%%%%Master
\subsection{\no{Student}}
\cvline{Q3.2013 - Q2.2014}{NTNU}
\cvline{}{
I sin masteroppgave har Magnus undersøkt tre ting. Hvordan man kan klassifisere
sentimentet til en tweet, om twitter-data kan bli brukt til å aggregere
trender, og hvordan de twitter-aggregerte trendene stemmer overens med teknisk
analyse i aksjemarkedet. Avhandlingen konkludere med at det er mulig å benytte
twitterdata i algoritmer for teknisk analyse, men at man foreløpig vet for lite om hvor
gode tilnærmingene faktisk er.  
}
%%-----------------------------------------------------------------------------
% Technologies used in this project.  
\cvline{Teknikker}{
Python, Flask, \LaTeX, nltk,  
}
%%-----------------------------------------------------------------------------
% Environments used in this project. 
\cvline{Miljøer}{
Linux, Pycharm,
}
%%-----------------------------------------------------------------------------

%%%%SM - Hybel
%\subsection{\no{Intern Områdeansvarlig}}
%\cvline{Q2.2013 - Q1.2014}{Studentmediene i Trondheim AS}
%\cvline{}{
%}
%%-----------------------------------------------------------------------------
% Technologies used in this project.  
%\cvline{Teknikker}{
%}
%%-----------------------------------------------------------------------------
% Environments used in this project. 
%\cvline{Miljøer}{
%}
%%-----------------------------------------------------------------------------

%%%%SM - IT-sjef
\subsection{\no{IT-Sjef}}
\cvline{Q2.2013 - Q2.2014}{Studentmediene i Trondheim AS}
\cvline{}{
Som IT-sjef har Magnus tatt del i selskapets ledergruppe. Ledergruppa er
sammensatt av administrativ, redaksjonell og teknisk ledelse. Ledergruppa har
som oppgave å ta overordnede avgjørelser relatert til økonomi, administrasjon,
og strategi, samt ta seg av den daglige driften av selskapet.    
\newline
IT-avdelingen var hovedansvaret til Magnus. IT-avdelingen består av 25 utviklere
og prosjekledere som Magnus har hatt personalansvar for. I sin periode som
IT-sjef bygde Magnus opp avdelingen på nytt. Dette var i sammenheng med
omorganiseringen av studentmediene i Trondheim. Gjennom oppbyggingen av
avdelingen har det været fokus på å bygge kultur og sosialt samhold.  
\newline
Magnus har hatt ansvaret for alle IT-utviklingsprosjekter i selskapet. Dette er
i hovedsak syv prosjekter; Anthropos, Barteguiden, Barweb, Chimera, iBok,
InternSida, Momus. Som del av prosjektansvaret er veiledning og opplæring av
medarbeiderene. 
\newline
Magnus har laget egen strategi for IT-utvikling. Strategien tar for seg hvilke
områder man bør satse på, hvilken rolle IT-avdelingen har i resten av
organisasjonen, hvilke prosjekter og teknologier man bør satse på, og hvordan
man kan utvide tjenestetilbudet innad i organisasjonen.
\newline
Da Magnus var ferdig i stillingen hadde IT-avdelingen fått en god og fungerende
struktur med godt miljø og god kultur. Magnus har også klart å bygge kompetanse i
perioden, noe man nyter godt av i etterkant. Magnus har i helhet lagt et godt
grunnlag for videre utvikling av IT-avdelingen og de IT-prosjekter som
foreligger i Studentmediene. 
}
%%-----------------------------------------------------------------------------
% Technologies used in this project.  
\cvline{Teknikker}{
Java, HTML5, CSS3, JavaScript, Spring,
}
%%-----------------------------------------------------------------------------
% Environments used in this project. 
\cvline{Miljøer}{
Linux, Ubuntu Server, Intellij IDEA,
}
%%-----------------------------------------------------------------------------

%%%%Academic Work / Coperio
\subsection{\no{Konsulent, Webutvikler, IT-support}}
\cvline{Q4.2012 - Q2.2013}{Academic Work}
\cvline{}{
På oppdrag hos Coperio jobbet Magnus med IT-support og webutvikling. Han hjalp
ansatte med problemer på Windows 7, oppsett av nye maskiner, printer oppsett, og MS
Office.
Utviklingen bestod i å bytte ut Coperio sin nettside med en ny og bedre side
bassert på Joomla. Hovedarbeidet bestod i oppsett av Joomla, og utforming av
utseendet på nettsiden.  
}
%%-----------------------------------------------------------------------------
% Technologies used in this project.  
\cvline{Teknikker}{
PHP, HTML5, CSS3, JavaScript, MySQL
}
%%-----------------------------------------------------------------------------
% Environments used in this project. 
\cvline{Miljøer}{
Linux, Intellij IDEA, phpMyAdmin, Joomla,
Windows 7, MS Outlook, MS Office,
}
%%-----------------------------------------------------------------------------

%%%%UD-hybel
%\subsection{\no{Intern Områdeansvarlig}}
%\cvline{Q3.2014 - Q2.2015}{Under Dusken}
%\cvline{}{
%Utvikling av nettside + Detljert prosjektbeskrivelse
%}
%%-----------------------------------------------------------------------------
% Technologies used in this project.  
%\cvline{Teknikker}{
%}
%%-----------------------------------------------------------------------------
% Environments used in this project. 
%\cvline{Miljøer}{
%}
%%-----------------------------------------------------------------------------

%%%%iNic AS
\subsection{\no{Systemutvikler}}
\cvline{Q2.2012 - Q3.2012}{iNic AS}
\cvline{}{
Magnus startet utviklingen av en web-service i Perl som registrerte domener opp
mot
systemene til NORID. Utviklingen ble utført i et Linux miljø. Vim ble brukt som
editor. Bash/shell, til testing og feilsøking. MySQL som test database. Git som
versjonskontrollsystem for koden. I løpet av sommerjobben var målet å få
registrert domener hos NORID, noe som fungerte på slutten av engasjementet.
\newline
Han satte opp og innførte Git, versjonskontroll av kildekode,
internt i selskapet.
}
%%-----------------------------------------------------------------------------
% Technologies used in this project. Languages and frameworks. 
\cvline{Teknikker}{
Perl, XML, Bash, Shell, Net::EPP::Client
}
%%-----------------------------------------------------------------------------
% Environments used in this project. Platforms and products.  
\cvline{Miljøer}{
Linux, Ubuntu Server, Vim, Git, MySQL
}
%%-----------------------------------------------------------------------------

%%%%UD-Maskinist
\subsection{\no{Maskinist}}
\cvline{Q3.2010 - Q2.2012}{Under Dusken}
\cvline{}{
Mangus har vært en del av styret i Under Dusken, samlingen av redaktører og
andre ledere som drev avisa. Maskinist stillingen har personalansvaret for
de 6 datamedarbeiderene man hadde i Under Dusken.   
\newline
%systemutviklnig og prosjektansvar(Aranea, Pegadi, Annonseweb, Barweb)
En sentral oppgave for Maskinisten er å drive utvikling av avisa sine
IT-systemer. Dette innebar prosjektansvar, veiledning, og utvikling av
systemene. De to største prosjektene var Aranea og Pegadi. Aranea var
applikasjonen som kjørte nettsidene til Under Dusken. Pegadi var et innhold-håndteringssystem 
for artikler til papirutgaven av avisa.   
\newline
Som Maskinist har Magnus hatt ansvaret for alt som har med IT-drift å gjøre.
Det omfatter servere, arbeidsstasjoner, nettverk og printere. I all hovedsak
alt av infrastruktur en avisredaksjon kan trenge.  
}
%%-----------------------------------------------------------------------------
% Technologies used in this project.  
\cvline{Teknikker}{
Java, HTML, CSS, JavaScript, Spring, MySQL, PHP, LDAP, Apache VelocityView, XML 
}
%%-----------------------------------------------------------------------------
% Environments used in this project. 
\cvline{Miljøer}{
Linux, Ubuntu Server, Debian, Intellij IDEA, phpMyAdmin, Apache2, Windows 7,
Jenkins, phpLDAPAdmin, Sonatype Nexsus, MediaWiki 
}
%%-----------------------------------------------------------------------------

%%%%Bachelor
\subsection{\no{Student}}
\cvline{Q1.2012 - Q2.2012}{NTNU, Forsvarets Forskningsinstitutt}
\cvline{}{
Magnus har gjennom bachelor oppgaven i informatikk laget et 'proof of concept'
for FFI. Han var med på å undersøke om man kan prioritere nettverkstrafikken
til web-services i militære nettverk. Resultatene av oppgaven ble publisert av
FFI i Globecom Workshops.  
}
%%-----------------------------------------------------------------------------
% Technologies used in this project.  
\cvline{Teknikker}{
\LaTeX, Axiom 1.2.11, Commons-logging 1.1.1, 
Git 1.7.x, HTTPCore 4.1.4, JUnit 4.x, MobiEmu, ns-3 3.13, 
OpenJDK Java 1.6.023, WSO2 ESB 4.0.3.
}
%%-----------------------------------------------------------------------------
% Environments used in this project. 
\cvline{Miljøer}{
Linux, Vim, Eclipse Indigo 3.7.x, GlassFish 3.1.1
}
%%-----------------------------------------------------------------------------

%%%%UD-Aranea
\subsection{\no{Systemutvikler}}
\cvline{Q1.2010 - Q2.2010}{Under Dusken}
\cvline{}{
Magnus deltok i videreutviklingen av Aranea, prosjektet for nettsiden til Under
Dusken. Han har jobbet med utvikling av ny funksjonalitet, og vedlikehold av
tjenesten.
}
%%-----------------------------------------------------------------------------
% Technologies used in this project.  
\cvline{Teknikker}{
Java, HTML, CSS, Spring, Apache VelocityView, XML, MYSQL
}
%%-----------------------------------------------------------------------------
% Environments used in this project. 
\cvline{Miljøer}{
Linux, Ubuntu Server, Intellij IDEA,
}
%%-----------------------------------------------------------------------------


% Highlights from projects andpossitions
%\section{\no{Prosjekterfaring (Fremhevede prosjekter)}\en{Project Experience
%(Highlights)}}
%- trekke fram de prosjektene som har vært viktigst. 


\end{document}
