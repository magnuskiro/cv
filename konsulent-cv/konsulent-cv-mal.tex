\documentclass[11pt,a4paper,sans]{moderncv}

%%-----------------------------------------------------------------------------
%% Required package, everything inside this is for Norwegian/English language
%%support
\usepackage{ifthen}
\usepackage[norsk,english]{babel}

%% TeX function that generates the language commands.
\def\localedef#1#2{
\ifthenelse{ \equal{\locale}{#1} }{
  \selectlanguage{#2}
  \expandafter\def\csname#1\endcsname ##1{##1}
  }{
  \expandafter\def\csname#1\endcsname ##1{}
  }
}

%%Controls the language output
\providecommand\locale{no}

%% Language commands.
\localedef{no}{norsk}
\localedef{en}{english}
%%-----------------------------------------------------------------------------

%% ModernCV themes
\moderncvtheme[green]{classic}
%\moderncvtheme[black]{banking}
\renewcommand{\familydefault}{\sfdefault}
%\nopagenumbers{}

%% Character encoding
\usepackage[utf8]{inputenc}

%% Adjust the page margins
\usepackage[scale=0.90]{geometry}

%%URL support
\usepackage{hyperref}

%%% Classic style edits. 
%%% http://ctan.uib.no/macros/latex/contrib/moderncv/moderncvstyleclassic.sty
%% change size of contact info. 
% fonts
\renewcommand*{\addressfont}{\large\mdseries\slshape}
% styles
\renewcommand*{\addressstyle}[1]{{\addressfont\textcolor{color1}{#1}}}
%% set size of left column. 
\setlength{\hintscolumnwidth}{3.1cm}

%% Personal data
\firstname{Magnus Løken}
\familyname{Kirø}
%\address{Postboks 146, 
%\en{Apartment}\no{Leilighet} 
%}{3166 Tolvsrød,
%\en{Norway}\no{Norge}}
%\phone[mobile]{+47~91~74~85~66}
%\email{magnuskiro@gmail.com}
%\homepage{magnuskiro.no}
%\social[linkedin]{magnuskiro}
%\social[twitter]{magnuskiro}
%\social[github]{magnuskiro}
\extrainfo{\textdagger 15.03.1990}
%\photo[100pt][1pt]{test.jpg}

% title is the current position you have. 
\title{Web-/Systemutvikler (bruk en eller flere titler)}
%%------------------------------------------------------------------------------
%% Content
%%------------------------------------------------------------------------------
\begin{document}
\makecvtitle{}

% Sammendrag av deg som kosnulent.
\no{
Her skal du skrive en egen beskrivelse av deg. 

1. Innledning: Beskriv din kompetanse  med to-tre setninger, referere til din
erfaring og  kompetens, bransch eller projekt. 

2. Utdanning:  Utdannet innen… retning, eventuelle sertifisernger..

3. Hva utmerker deg/ Hvordan fungerer du i prosjekt?

 ”Det som utmerker seg ved Anna i prosjekt er..”  hvorfan jobber du i prosjekt,
lederrolle, drivkraft, problemløser, etc …”.

Bruk malen som inspirasjon og skriv en egen presentasjon ut i fra    bredvid som
inspiration och skriv en egen presentation utifra samme stil.
}

% Beskrivende hovedtrekk ved deg.
\subsection{\no{Hovedkvalifikasjoner /
Spesialkunnskaper}}\label{specialQualifications}
\no{
Her skriver du dine fremste kompetanseområder. 
\newline
\newline
\cvline{Fremhevet Erfaring}{
\begin{itemize}
\item  Løft frem tre erfaringer du er mest stolt av som karakteriserer deg litt
ekstra. (punktliste)
\item b
\item c
\end{itemize}
}
}

\section{\no{Web-/Systemutvikler (samme tittel som øverst på første side.)}}
\subsection{\no{Teknisk Profil}}
 Under denne seksjonen og de følgende skal du liste samtlige teknikker du har
 jobbet med, uansett hvor stor eller liten erfarening du har innen aktuell
 teknikk. Ikke vær redd for å ta med noe selv om du bare har sittet og plukket
 litt med teknikken. Dersom du har jobbet med en teknikk er det kunnskap,
 uansett om du er ekspert innen emnet eller ei. Det kreves ikke at du fått
betalt for å lære deg noe. Kompetanse som kommer fra egenintresse er lik
relevant.
\newline

\cvline{Arbeidsområde}{
Webutvikling, Systemutvikling, Systemintegrasjon, Kravspessifisering,
Systemarkitektur, Databasedesign, Personlaledelse, Prosjektledelse, Organisasjonsledelse
}
%%-----------------------------------------------------------------------------
\cvline{Utviklingsspråk og rammeverk}{
Java, Pythnon, PHP, C, C++, GO, Perl, Bash, Shell, JavaScript, CSS3, HTML5,
Assembly, 
Spring, Flask, jQuery, Knockout.js, Angular.js,  
}
%%-----------------------------------------------------------------------------
\cvline{Web- og mobilutvikling}{
XML, HTML, XHTML, JSON, Web Services
}
%%-----------------------------------------------------------------------------
\cvline{Arkitektur og Design}{
BPMN, UML, MVC, ITIL Foundation, 
}
%%-----------------------------------------------------------------------------
\cvline{Metodikk}{
Agile, Scrum, Waterfall, 
}
%%-----------------------------------------------------------------------------
\cvline{Databaser}{
PostgreSQL, mySQL
}
%%-----------------------------------------------------------------------------
\cvline{Test}{
Junit, Nunit
}
%%-----------------------------------------------------------------------------
\cvline{Produkter/miljøer}{
Intellij IDEA, Pycharm, Linux, Linux Mint, Debian, Ubuntu, Confluence, Android SDK, \latex,
Google analytics, SSH, Subversion, Git, Eclipse, GIMP, Vim,   
}
%%-----------------------------------------------------------------------------
\cvline{Annet}{
Ledelse, Økonomi, etc annet stuffs. 
}
%%-----------------------------------------------------------------------------
% Make sure this section is the same as \ref{specialQualifications}
\cvline{Spesialkunnskaper}{
Systemutvikling, prosjketoppfølging, prosjektledelse, 
}
%%-----------------------------------------------------------------------------

\subsection{\no{Annet}}
[- Her skal du skrive alle relevante utdanninger og sertifiseringer du har tatt.
Samt språk du kan.]
\newline

% copy from normal cv. 
\cvline{Utdanning}{
- Master i Informatikk fra Norges Tekniske- og Nturvitenskaplige Universitet(NTNU), med
spessialisering innenfor informasjonsforvaltning.
\newline
- Bachelor
\newline
-  
}
%%-----------------------------------------------------------------------------
\cvline{Kurs of Sertefiseringer}{
- Regnskapskurs for frivillige organisasjoner etc. 
}
%%-----------------------------------------------------------------------------
\cvline{Språk}{
Norsk, Engelsk, Tysk
}
%%-----------------------------------------------------------------------------

% Separation here summary first, experience after this. 
%\newpage

% list / table of employees
\section{\no{Arbeidserfaring}\en{Work Experience}}
[- Tabell med oversikt over arbeidsgivere og år for ansettelse.]

\subsection{\no{Arbeidsgivers navn}}
[- Beskrivelse av gitt arbeidsgiver. Hvem, hva, hvorfor.]
[- Her under lister du opp all prosjekterfaring. Ikke glem deltidsstillinger og
ekstrajobber. 
Del opp lengre prosjekt eller stillinger etter ulike prosjekt du jobbet  i
og/eller etter ulike roller du hatt i prosjektet slik at det blir flere poster,
da blir din kompetanse mer lettlest.]

% detailed descriptions of positions and projects.
\section{\no{Prosjekterfaring}\en{Qualifications}}
[- Detaljerte beskrivelser av arbeidet som er utført hos de forskjellige
arbeidsgiverene.]

[- Her under lister du opp all prosjekterfaring. Ikke glem deltidsstillinger og
ekstrajobber. 
Del opp lengre prosjekt eller stillinger etter ulike prosjekt du jobbet  i
og/eller etter ulike roller du hatt i prosjektet slik at det blir flere poster,
da blir din kompetanse mer lettlest.]

	% position
\subsection{\no{Stillingstittel}}
Stillingstittelen skal beskrive din rolle/dine roller i prosjektet. Bruk gjerne
flere titler i rubrikken for å vise hele prosjektet og ansvar du hadde utover
hovedansvaret, f.eks.”Systemutvikler, testleder”.

% The time period and the employer. 
\cvline{Q3 2014 - dd}{Arbeidsgiver
\newline
[- <-- Datoformatet til venstre skal være kvartalsbassert, Q1-4. Prosjekter som er
kortere enn et kvartal skal plasseres i det kvartalet prosjektet varte.]
\newline
[- Navnet på arbeidsgiver skal være på 1(en) linje.]
}
%%-----------------------------------------------------------------------------
% description of this position. 
\cvline{}{
Beskrivelsen av prosjektet: 
\newline
[- Denne teksten skal utgå fra problemstillingene: Hva gjør selskapet?, Hva var
problemet/hva trengte de hjelp med? Hvordan hjalpe jeg dem/hvordan løste jeg
problemet? Hva ledet det til/hvordan endte prosjektet. Gjerne en
suksesshistorie.]
}
%%-----------------------------------------------------------------------------
% Technologies used in this project.  
\cvline{Teknologier}{
Java, HTML5, CSS3, JavaScript, Spring
\newline
[- I rubrikken “Teknologier” skal du liste hvilke ulike teknikker du anvendte under
aktuelt prosjekt. Dette kan være elementer som programmeringsspråk (C++/Java),
markup-languages (HTML/XML) og liknende teknikker.]
}
%%-----------------------------------------------------------------------------
% Environments used in this project. 
\cvline{Miljøer}{
Linux, Ubuntu Server, Debian, Linux Mint, Intellij IDEA, 
\newline
[- I rubrikken ”Miljøer” skal du
liste de verktøyene som du anvender for å utvikle med disse teknikkene. Miljøer
kan både være operativsystem ( Windows/Linux/Unix/MacOS) men også
programmeringsmiljøer som Visual Studio, Eclipse, JBoss og så videre.]
}
%%-----------------------------------------------------------------------------

    % position
\subsection{\no{Systemutvikler}}
% description of this position. 
\cvline{Q3.2014 - Q2.2015}{Næringslivsnettverket KID}
\cvline{}{
Utvikling av nettside + Detljert prosjektbeskrivelse
}
%%-----------------------------------------------------------------------------
% Technologies used in this project.  
\cvline{Technologier}{
Java, HTML5, CSS3, JavaScript, Spring
}
%%-----------------------------------------------------------------------------
% Environments used in this project. 
\cvline{Miljøer}{
Linux, Ubuntu Server, Debian, Linux Mint, Intellij IDEA,
}
%%-----------------------------------------------------------------------------

% Highlights from projects andpossitions
\section{\no{Prosjekterfaring (Fremhevede prosjekter)}\en{Project Experience
(Highlights)}}
- trekke fram de prosjektene som har vært viktigst. 


\end{document}
