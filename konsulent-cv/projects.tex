

%%%%KID
\subsection{\no{Systemutvikler}}
% description of this position. 
\cvline{Q3.2014 - Q2.2015}{Næringslivsnettverket KID}
\cvline{}{
Magnus har utviklet http://kid.item.ntnu.no/ videre. Han har jobbet mest med backend
programmernig og drift av webserveren. Magnus har også
jobbet med frontend-utvikling. 
}
%%-----------------------------------------------------------------------------
% Technologies used in this project.  
\cvline{Teknikker}{
Java, Scala, HTML5, CSS3, JavaScript, PlayFramework, Knockout.js, jQuery, MySQL 
}
%%-----------------------------------------------------------------------------
% Environments used in this project. 
\cvline{Miljøer}{
Linux, Ubuntu Server, Intellij IDEA, phpMyAdmin, 
}
%%-----------------------------------------------------------------------------

%%%%SM - Barweb
\subsection{\no{Systemutvikler}}
\cvline{Q3.2014 - Q4.2014}{Studentmediene i Trondheim AS}
\cvline{}{
Magnus har startet ny-utviklingen av et internt system fra Under Dusken og
fullførte alpha versjonen.
Systemet holder oversikt over transaksjoner og regninger, fra sosiale
sammenkomster på Samfundet.
Prosjektet ble overlevert ny prosjekleder i god
stand, og med klare mål og oppgaver for videre utvikling.
}
%%-----------------------------------------------------------------------------
% Technologies used in this project.  
\cvline{Teknikker}{
Java, HTML5, CSS3, JavaScript, Spring, Angular.js, XML
}
%%-----------------------------------------------------------------------------
% Environments used in this project. 
\cvline{Miljøer}{
Linux, Ubuntu Server, Intellij IDEA, Jenkins
}
%%-----------------------------------------------------------------------------

%%%%SM - IT-drift
\subsection{\no{IT-drift}}
\cvline{Q2.2013 - Q4.2014}{Studentmediene i Trondheim AS}
\cvline{}{
Magnus har gjennom sin tid i Studentmediene vært med i IT-drifts-komiteen.
Komiteen har ansvaret for servere, arbeidsstasjoner, nettverk, og printere.
IT-drift jobber til daglig tett opp mot IT-utvikling. Magnus har hatt ansvar
for de gamle serveren til Under Dusken og utfasing av disse og han har hatt en
rådgivende rolle i komiteen. Magnus har tatt del i oppsett og planlegging av ny
infrastruktur.
}
%%-----------------------------------------------------------------------------
% Technologies used in this project.  
\cvline{Teknikker}{
}
%%-----------------------------------------------------------------------------
% Environments used in this project. 
\cvline{Miljøer}{
Linux, Ubuntu Server, Debian, Intellij IDEA, MediaWiki,
}
%%-----------------------------------------------------------------------------

%%%%Master
\subsection{\no{Student}}
\cvline{Q3.2013 - Q2.2014}{NTNU}
\cvline{}{
I sin masteroppgave har Magnus undersøkt tre ting. Hvordan man kan klassifisere
sentimentet til en tweet, om twitter-data kan bli brukt til å aggregere
trender, og hvordan de twitter-aggregerte trendene stemmer overens med teknisk
analyse i aksjemarkedet. Avhandlingen konkludere med at det er mulig å benytte
twitterdata i algoritmer for teknisk analyse, men at man foreløpig vet for lite om hvor
gode tilnærmingene faktisk er.  
}
%%-----------------------------------------------------------------------------
% Technologies used in this project.  
\cvline{Teknikker}{
Python, Flask, \LaTeX, nltk,  
}
%%-----------------------------------------------------------------------------
% Environments used in this project. 
\cvline{Miljøer}{
Linux, Pycharm,
}
%%-----------------------------------------------------------------------------

%%%%SM - Hybel
%\subsection{\no{Intern Områdeansvarlig}}
%\cvline{Q2.2013 - Q1.2014}{Studentmediene i Trondheim AS}
%\cvline{}{
%}
%%-----------------------------------------------------------------------------
% Technologies used in this project.  
%\cvline{Teknikker}{
%}
%%-----------------------------------------------------------------------------
% Environments used in this project. 
%\cvline{Miljøer}{
%}
%%-----------------------------------------------------------------------------

%%%%SM - IT-sjef
\subsection{\no{IT-Sjef}}
\cvline{Q2.2013 - Q2.2014}{Studentmediene i Trondheim AS}
\cvline{}{
Som IT-sjef har Magnus tatt del i selskapets ledergruppe. Ledergruppa er
sammensatt av administrativ, redaksjonell og teknisk ledelse. Ledergruppa har
som oppgave å ta overordnede avgjørelser relatert til økonomi, administrasjon,
og strategi, samt ta seg av den daglige driften av selskapet.    
\newline
IT-avdelingen var hovedansvaret til Magnus. IT-avdelingen består av 25 utviklere
og prosjekledere som Magnus har hatt personalansvar for. I sin periode som
IT-sjef bygde Magnus opp avdelingen på nytt. Dette var i sammenheng med
omorganiseringen av studentmediene i Trondheim. Gjennom oppbyggingen av
avdelingen har det været fokus på å bygge kultur og sosialt samhold.  
\newline
Magnus har hatt ansvaret for alle IT-utviklingsprosjekter i selskapet. Dette er
i hovedsak syv prosjekter; Anthropos, Barteguiden, Barweb, Chimera, iBok,
InternSida, Momus. Som del av prosjektansvaret er veiledning og opplæring av
medarbeiderene. 
\newline
Magnus har laget egen strategi for IT-utvikling. Strategien tar for seg hvilke
områder man bør satse på, hvilken rolle IT-avdelingen har i resten av
organisasjonen, hvilke prosjekter og teknologier man bør satse på, og hvordan
man kan utvide tjenestetilbudet innad i organisasjonen.
\newline
Da Magnus var ferdig i stillingen hadde IT-avdelingen fått en god og fungerende
struktur med godt miljø og god kultur. Magnus har også klart å bygge kompetanse i
perioden, noe man nyter godt av i etterkant. Magnus har i helhet lagt et godt
grunnlag for videre utvikling av IT-avdelingen og de IT-prosjekter som
foreligger i Studentmediene. 
}
%%-----------------------------------------------------------------------------
% Technologies used in this project.  
\cvline{Teknikker}{
Java, HTML5, CSS3, JavaScript, Spring,
}
%%-----------------------------------------------------------------------------
% Environments used in this project. 
\cvline{Miljøer}{
Linux, Ubuntu Server, Intellij IDEA,
}
%%-----------------------------------------------------------------------------

%%%%Academic Work / Coperio
\subsection{\no{Konsulent, Webutvikler, IT-support}}
\cvline{Q4.2012 - Q2.2013}{Academic Work}
\cvline{}{
På oppdrag hos Coperio jobbet Magnus med IT-support og webutvikling. Han hjalp
ansatte med problemer på Windows 7, oppsett av nye maskiner, printer oppsett, og MS
Office.
Utviklingen bestod i å bytte ut Coperio sin nettside med en ny og bedre side
bassert på Joomla. Hovedarbeidet bestod i oppsett av Joomla, og utforming av
utseendet på nettsiden.  
}
%%-----------------------------------------------------------------------------
% Technologies used in this project.  
\cvline{Teknikker}{
PHP, HTML5, CSS3, JavaScript, MySQL
}
%%-----------------------------------------------------------------------------
% Environments used in this project. 
\cvline{Miljøer}{
Linux, Intellij IDEA, phpMyAdmin, Joomla,
Windows 7, MS Outlook, MS Office,
}
%%-----------------------------------------------------------------------------

%%%%UD-hybel
%\subsection{\no{Intern Områdeansvarlig}}
%\cvline{Q3.2014 - Q2.2015}{Under Dusken}
%\cvline{}{
%Utvikling av nettside + Detljert prosjektbeskrivelse
%}
%%-----------------------------------------------------------------------------
% Technologies used in this project.  
%\cvline{Teknikker}{
%}
%%-----------------------------------------------------------------------------
% Environments used in this project. 
%\cvline{Miljøer}{
%}
%%-----------------------------------------------------------------------------

%%%%iNic AS
\subsection{\no{Systemutvikler}}
\cvline{Q2.2012 - Q3.2012}{iNic AS}
\cvline{}{
Magnus startet utviklingen av en web-service i Perl som registrerte domener opp
mot
systemene til NORID. Utviklingen ble utført i et Linux miljø. Vim ble brukt som
editor. Bash/shell, til testing og feilsøking. MySQL som test database. Git som
versjonskontrollsystem for koden. I løpet av sommerjobben var målet å få
registrert domener hos NORID, noe som fungerte på slutten av engasjementet.
\newline
Han satte opp og innførte Git, versjonskontroll av kildekode,
internt i selskapet.
}
%%-----------------------------------------------------------------------------
% Technologies used in this project. Languages and frameworks. 
\cvline{Teknikker}{
Perl, XML, Bash, Shell, Net::EPP::Client
}
%%-----------------------------------------------------------------------------
% Environments used in this project. Platforms and products.  
\cvline{Miljøer}{
Linux, Ubuntu Server, Vim, Git, MySQL
}
%%-----------------------------------------------------------------------------

%%%%UD-Maskinist
\subsection{\no{Maskinist}}
\cvline{Q3.2010 - Q2.2012}{Under Dusken}
\cvline{}{
Mangus har vært en del av styret i Under Dusken, samlingen av redaktører og
andre ledere som drev avisa. Maskinist stillingen har personalansvaret for
de 6 datamedarbeiderene man hadde i Under Dusken.   
\newline
%systemutviklnig og prosjektansvar(Aranea, Pegadi, Annonseweb, Barweb)
En sentral oppgave for Maskinisten er å drive utvikling av avisa sine
IT-systemer. Dette innebar prosjektansvar, veiledning, og utvikling av
systemene. De to største prosjektene var Aranea og Pegadi. Aranea var
applikasjonen som kjørte nettsidene til Under Dusken. Pegadi var et innhold-håndteringssystem 
for artikler til papirutgaven av avisa.   
\newline
Som Maskinist har Magnus hatt ansvaret for alt som har med IT-drift å gjøre.
Det omfatter servere, arbeidsstasjoner, nettverk og printere. I all hovedsak
alt av infrastruktur en avisredaksjon kan trenge.  
}
%%-----------------------------------------------------------------------------
% Technologies used in this project.  
\cvline{Teknikker}{
Java, HTML, CSS, JavaScript, Spring, MySQL, PHP, LDAP, Apache VelocityView, XML 
}
%%-----------------------------------------------------------------------------
% Environments used in this project. 
\cvline{Miljøer}{
Linux, Ubuntu Server, Debian, Intellij IDEA, phpMyAdmin, Apache2, Windows 7,
Jenkins, phpLDAPAdmin, Sonatype Nexsus, MediaWiki 
}
%%-----------------------------------------------------------------------------

%%%%Bachelor
\subsection{\no{Student}}
\cvline{Q1.2012 - Q2.2012}{NTNU, Forsvarets Forskningsinstitutt}
\cvline{}{
Magnus har gjennom bachelor oppgaven i informatikk laget et 'proof of concept'
for FFI. Han var med på å undersøke om man kan prioritere nettverkstrafikken
til web-services i militære nettverk. Resultatene av oppgaven ble publisert av
FFI i Globecom Workshops.  
}
%%-----------------------------------------------------------------------------
% Technologies used in this project.  
\cvline{Teknikker}{
\LaTeX, Axiom 1.2.11, Commons-logging 1.1.1, 
Git 1.7.x, HTTPCore 4.1.4, JUnit 4.x, MobiEmu, ns-3 3.13, 
OpenJDK Java 1.6.023, WSO2 ESB 4.0.3.
}
%%-----------------------------------------------------------------------------
% Environments used in this project. 
\cvline{Miljøer}{
Linux, Vim, Eclipse Indigo 3.7.x, GlassFish 3.1.1
}
%%-----------------------------------------------------------------------------

%%%%UD-Aranea
\subsection{\no{Systemutvikler}}
\cvline{Q1.2010 - Q2.2010}{Under Dusken}
\cvline{}{
Magnus deltok i videreutviklingen av Aranea, prosjektet for nettsiden til Under
Dusken. Han har jobbet med utvikling av ny funksjonalitet, og vedlikehold av
tjenesten.
}
%%-----------------------------------------------------------------------------
% Technologies used in this project.  
\cvline{Teknikker}{
Java, HTML, CSS, Spring, Apache VelocityView, XML, MYSQL
}
%%-----------------------------------------------------------------------------
% Environments used in this project. 
\cvline{Miljøer}{
Linux, Ubuntu Server, Intellij IDEA,
}
%%-----------------------------------------------------------------------------
